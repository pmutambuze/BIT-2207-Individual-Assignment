\documentclass[options]{article}

 \usepackage[
    top    = 2.75cm,
    bottom = 2.50cm,
    left   = 4.00cm,
    right  = 3.50cm]{geometry}

\usepackage[parfill]{parskip}
\pagenumbering{roman}
\title{SMART LOCK FOR PASSWORD ON ANDROID  (Case Study Smart Home)}
\author{Mutambuze Paul  16/U/7738/EVE  216012181\thanks{Lecturer: Dr. Ernest Mwebaze}}\newpage
\date{%
    Makerere University\\%
    March 9, 2018
}


\begin{document}
\begin{titlepage}
\maketitle
\end{titlepage}

\newpage
\pagenumbering{arabic} 
\section{\textbf{Literature Review}}
Various control systems have been designed over the years to prevent access to unauthorized users.
The main aim for providing locking systems for our various accounts, home, office and building is for security of
our private files, lives and property. It is therefore important to have convenient way of achieving 
this goal. Today, most mobile phones are a 'smart phone running Android OS', which offers more 
advanced capabilities in connectivity issues than regular cell phones \cite{key:foo}. According to an investigate by ABI Research, at the end of 2013, 1.4 billion smart phones has been
  in use: 798 million of them run Android, 294 million run Apple’s iOS, and 45 million run Windows Phone \cite{foo:baz}. 
  
\cite{key:foo}This document gives overall idea of how to control your security with \textbf{ Google’s Smart Lock for Passwords} for door key locks.
They use android based door lock system for indoor and outdoor key lock system. It also provides a secure system for Android
phone users. This document is based on Android platform which is Free Open Source i.e. it is easily available. So the implementation
rate is inexpensive and it is reasonable for a common person. The wireless connection in microcontroller permits the system
installation in an easier way. The system has been designed successfully and aimed to control the door condition using an
Android phone which is wireless enabled.


\cite{key:lee}This document further gives detail information about system in which we can unlock the door by using pre-decided password.
It increases the security level to prevent an unauthorized unlocking done by attackers. In case the user forgets the both
passwords, this Google’s smart lock gives the flexibility to the user to change or reset the password. This automatic password
based lock system will give user a more secure way of locking-unlocking system. First the user combination will be compared
with prerecorded password which are stored in the system memory. User can go for certain number of wrong combinations
before the system will be temporarily disabled. The door will be unlocked if user combination matches with the password.
The same password can be used to lock the door as well. This system will give the user an opportunity to reset his own
password if he wants.


\cite{lee:pak}The document proposed idea that in day to day life security of any object or place password-based system plays a major role.
This paper has considered about this and created a secure access for a door which needs a password to unlock the door. Using
keypad, it enters a password to the system and if entered password is correct then door is open by motor which is used to
rotate the handle of the door lock. When it is entered incorrectly at the first time it will give three attempts to enter the 
password.

\cite{key:mut}This document also gives basic idea of how to control various home appliances and provide a security using Android 
phone/tab. This project is based on Android platform which is FOSS(Free Open Source Software). So the overall implementation
cost is very cheap and it is affordable by a common person.

\cite{lee:kal}This paper proposed information that Smart phones usually support wireless technologies making it possible to 
transfer data through this connection. Smart phone can provide computer mobility, ubiquitous data access, and pervasive 
for almost every aspect of business processes and people’s daily lives. The next generation mobile computing will
 create a fantastic world in which we will be able to enjoy predecendent level of communication, computing and
  entertainment. 

The power of convergence of data access and pervasive mobile intelligence enabled by smart mobile devices such
as smart phones is the driving force behind this wave of computing. The first step to build a smart home is about 
the security and the door is the major device for security system using the Google Smart Lock for Passwords App.



\begin{thebibliography}{9}

\bibitem{key:foo}
Lia Kamelia, Alfin Noorhassan S.R, Mada Sanjaya.W.S, and Edi Mulyana, ”Door-Automation System
using Bluetooth-based Android for Mobile phone,” ARPN Journal of Engineering and Applied
Sciences vol. 9, no. 10, October 2014.

\bibitem{foo:baz}
2013. Business Insider Homepage [online], available at :http://www.businessinsider.com/15-billion-sm
rtphonesin- the-world-22013-2?IR=T.
 

\bibitem{key:lee}
Dibyendu Sur, Sayani Sengupta, Sarmistha Ray, Sucheta Routh, Saborni Das, Soumika Ghosh, Shilpi
Banerjee, “Digital password door lock security system “,National seminar on Advances of Security
Issues (ASICN 2013) July 19, 2013, DSCSDEC, Kolkata, ISBN No. 81-925299-4-0



\bibitem{lee:pak}
Arpita Mishra, Siddharth Sharma, Sachin Dubey, S.K.Dubey, “Password Based Security Lock
System”, International Journal of Advanced Technology in Engineering and Science, Volume No.02,
Issue No. 05, May 2014, ISSN (online): 2348 – 7550.


\bibitem{key:mut}
D. Javale, M. Mohsin, S. Nandanwar, and M. Shingate. Home Automation and Security System Using
Android ADK. International Journal of Electronics Communication and Computer Technology
(IJECCT) 2013. 3: 382-385.


\bibitem{lee:kal}
Pei Zheng, Lionel Ni. Smart Phone and Next Generation Mobile Computing, Morgan Kaufmann
publisher, san Fransisco 2006.



\end{thebibliography}

\end{document}